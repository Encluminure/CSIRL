\documentclass[a5paper,pagesize,DIV=14]{scrbook}

\usepackage[french]{babel}
\usepackage[utf8]{inputenc}
\usepackage[T1]{fontenc}
\usepackage{graphicx}
\usepackage{tabularx}
\usepackage{color}
\usepackage{tikz}
\usepackage{url}
\usepackage{chessboard}
\usepackage{import}
\usepackage{pdfpages,wrapfig}

\setlength{\parskip}{\smallskipamount}
\setlength{\parindent}{0pt}

%\input{tikz/baseball.tex}

\hyphenation{meil-leures}

% récupérer un peu de place au dessus des titres
\renewcommand{\chapterheadstartvskip}{\vspace *{-\baselineskip }} 

\newenvironment{encart}[2]
{
  \begin{center}
  \vbox{
  \colorbox{light-blue}{
  \begin{minipage}{\linewidth}
  \begin{center}\textbf{#1}\end{center}
  \end{minipage}}\\
  \colorbox{light-gray}{
  \begin{minipage}{\linewidth}
  #2
  \end{minipage}}
  }
  \end{center}
}

\title{Les Algorithmes} 
\date{}
%\author{author}


\AtBeginDocument{
  \def\labelitemi{$\bullet$}
  \def\labelitemii{$\bullet$}
  \def\labelitemiii{$\bullet$}
}

%\boldmath

\definecolor{light-gray}{gray}{0.95}
\definecolor{light-blue}{rgb}{0.7 0.7 0.9}

\begin{document}

%%%
%%% Couverture et ours
%%%

\includepdf[pages=1]{img/couverture.pdf}

\copyright\ 2011-2014 membres du projet SMN. Tous droits réservés.
  
"Sciences Manuelles du Numérique" est un \textbf{projet libre et ouvert}: vous
pouvez copier et modifier librement les ressources de ce projet sous les
conditions données par la CC-BY-SA (en bref, vous pouvez diffuser et modifier
ces ressources à condition que vous donniez les mêmes droits aux utilisateurs de
vos copies).

\bigskip

Ce projet est porté par l'association \textbf{Nancy Bidouille}
(\url{http://www.nybi.cc}), avec le soutien d'\textbf{Inria}
(\url{http://www.inria.fr}).
  
\bigskip

\begin{itemize}
\item Page web : \url{https://github.com/InfoSansOrdi}
\item Sources et ressources : \url{https://github.com/InfoSansOrdi}
\item Pour nous joindre : \url{infosansordi@inria.fr}
\end{itemize}

\bigskip
~\hfill\includegraphics[width=0.3\textwidth]{img/logo_nybicc.pdf}\hfill
\includegraphics[width=0.3\textwidth]{img/logo_inria.pdf}\hfill~

\vfill
\textbf{Crédits image}

{\footnotesize

P1: Chemin hamiltonien par Ch. Sommer (licence GFDL/CC-BY-SA)\\
\url{http://en.wikipedia.org/wiki/File:Hamiltonian_path.svg}

P\pageref{img:CSmajor}: Computer Science Major (licence CC-BY-NC)\\
\url{http://abstrusegoose.com/206}

% P\pageref{img:realprog}: Real programmers: \alert{Dessin non libre, à refaire.}
% \url{http://www.ninisworld.com/oddsends/justforfun/50realprogrammers.html}

%P\pageref{img:tsp}: exemple de TSP adapté de Wikipedia (licence GFDL/CC-BY-SA)\\
%\url{http://en.wikipedia.org/wiki/File:Aco_TSP.svg}
P\pageref{img:electric:city}: Electric City par Mathias M, licence CC-BY-SA
\url{http://www.flickr.com/photos/mathias_m/342535332/}
  
P\pageref{img:tsp_xkcd}: Travelling Salesman Problem par XKCD (licence CC BY-NC 2.5)\\
\url{http://xkcd.com/399/}
}

%%%
%%% Début des hostilités
%%%


\chapter*{La science informatique, sans ordinateur}

% Contrairement à ce que beaucoup de monde pense, les ordinateurs ne sont pas la
% seule raison d'être de l'informatique. Pour preuve, ce projet développe
% diverses activités à faire avec des pions, des jetons ou des bouts de bois,
% mais sans aucun ordinateur et même sans électricité. Pourtant, ces petits jeux
% permettront à chacun de découvrir de manière ludique les notions au cœur de
% l'informatique: ce qu'est un algorithme et qu'est ce qui fait qu'un algorithme
% est meilleur qu'un autre, ou encore comment coder et transmettre une
% information.

Trop souvent, lorsque l'on parle d'informatique, on pense à l'ordinateur utilisé
comme outil. L'informatique devient alors l'art d'utiliser l'ordinateur pour une
tâche donnée, ou de le réparer. Pourtant, dans les entrailles de cette machine
se cache un domaine scientifique très vaste, dont les ramifications dépassent
largement l'ordinateur et son fonctionnement.

Avec le projet \textbf{Sciences Manuelles du Numérique}, nous vous proposons
d'explorer la science informatique{\ldots} en retirant l'ordinateur ! Pour cela,
nous avons conçu une série d'activités ludiques introduisant des notions
fondamentales de l'informatique par le biais d'un support matériel, pour
\textit{apprendre avec les mains}. Ces activités sont regroupées en séances
thématiques, permettant ainsi d'aborder les notions fondamentales de manière
cohérente et progressive.

%\begin{block}{Introduction: les principales caractéristiques d'un ordinateur}
    %\begin{itemize}
    %\item \structure{Il est très \alert{rapide}:} il peut calculer de 1 à 1
    %  million en moins d'une seconde
    %\item \structure{Il est parfaitement \alert{obéissant}:} il fait
    %tout le temps exactement ce qu'on lui demande
    %\item \structure{Il est absolument \alert{stupide}:} il exécute les
      %ordres qu'on lui donne, sans la moindre capacité d'initiative.
      %\begin{itemize}
      %\item Par exemple, si on demande à un ordinateur de s'arrêter, il le fait{\ldots}
      %\item Autre exemple, quand j'indique à des amis comment venir chez moi,
        %je leur donne des indications comme ``troisième à droite'' ou ``à gauche
        %au 2ieme feu''. Si je me trompe dans mes indications (``à gauche'' au lieu
        %de ``à droite'') et que cela les ferait prendre l'autoroute à
        %contre-sens, mes amis vont faire preuve de sens commun et ne pas
        %appliquer la consigne. Les ordinateurs n'ont \textbf{aucun} sens commun.
        % \item Bug (n.m.): consigne erronée donnée par un humain et appliquée
        %   bêtement par une machine.
      %\end{itemize}
    %\end{itemize}
  %\end{block}\vspace{-.5\baselineskip}

\section*{La séance algorithmique}

L'objectif de cette séance est d'introduire la notion fondamentale
d'\textbf{al\-gorithme}. Qu'est-ce-que c'est ? A quoi ça sert ? Comment sont-ils
inventés ? Pour cette séance, comptez \textbf{une heure et demi à deux
  heures}. Bien entendu, il ne s'agit pas d'un cours complet sur le sujet ! Si
vous souhaitez aller plus loin, voici un cours en 48h :
\url{http://people.irisa.fr/Martin.Quinson/Teaching/TOP/}.

Les animateurs d'ateliers trouveront des conseils et astuces à la fin de chaque
activité dans \og le coin de l'animateur \fg ; mais les conseils les plus
importants sont certainement les suivants :

\begin{itemize}
\item appropriez-vous les activités. Pratiquez-les à l'avance et n'hésitez pas à
  ne pas suivre les consignes à la lettre;
\item ces activités sont des bases de discussion avec les participants, il n'y a
  pas d'évaluation à la fin;
\item évitez les introductions théoriques; commencez par les activités, elles
  serviront de support pour discuter de la théorie.
\end{itemize}

\begin{center}
  \includegraphics[width=0.7\linewidth]{img/computer_science_major.PNG}
  \label{img:CSmajor}
\end{center}

%%%%%%%%%%%%%%%%%%%%%%%%%%%%%%%%%%%%%%%%%%%%%%%%%%%%%%%%%%%%%%%%%%%%%%%%%
%%%%%%%%%%%%%%%%%%%%%%%%%%%%%%%%%%%%%%%%%%%%%%%%%%%%%%%%%%%%%%%%%%%%%%%%%
\chapter*{Le jeu de Nim}

Voici un premier petit jeu simple, pour rentrer dans le sujet. On dispose sur
une table \textit{16 objets}. Chacun leur tour, les deux joueurs ramassent
\textit{un, deux ou trois objets} sur la table. Le joueur qui \textbf{ramasse le
  dernier objet} remporte la partie.

\bigskip
 
\encart{Matériel}{
  \begin{itemize}
    \item 16 petits objets (clous, allumettes{\ldots}  peu importe !)
  \end{itemize}
}

\bigskip
\bigskip
\bigskip

\begin{center}
\input{nim/nim1.tex} \\
\textbf{Le joueur bleu gagne}
\end{center}

\newpage

\section*{Stratégie gagnante}

Le jeu de Nim est sans suspense : le premier à jouer perd, car il existe une
astuce pour que le deuxième joueur gagne à tous les coups. La \textbf{stratégie
  gagnante} est de laisser 4, 8, 12 ou 16 objets à l'adversaire (un multiple de
4).

Pour se convaincre de l'efficacité de la stratégie gagnante, prenons le dernier
tour comme exemple. Il reste 4 objets, et J1 joue :

\begin{itemize}
  \item si J1 prend 1 objet, J2 en prend 3 (dont le dernier) ;
  \item si J1 prend 2 objets, J2 en prend 2 (dont le dernier) ;
  \item si J1 prend 3 objets, J2 en prend 1 (le dernier).
\end{itemize}        

Dans ce cas, si J2 sait jouer, J1 perd à tous les coups. En appliquant la même
méthode, J2 peut guider le jeu de manière à passer de 16 objets à 12, puis 8 et
enfin 4. Donc, si J2 sait jouer, J1 a perdu la partie avant même de commencer.

\begin{center}
  \includegraphics[width=\linewidth]{nim/nim16.pdf}
\end{center}

\subsection*{Rapport avec l'informatique}

Comme pour le jeu de Nim, \textbf{un algorithme est une stratégie gagnante}
permettant de trouver la solution à un problème donné. Dans l'exemple précédent,
le problème était «~comment gagner au jeu de Nim ?~»

\encart{Pour aller plus loin}{
  On pourrait imaginer un cas plus général du jeu de Nim :

  \begin{itemize}
  \item Il y a $N$ objets sur la table au début du jeu \\ (pour notre version,
    $N=16$)
  \item Un joueur peut prendre jusqu'à $X$ objets à la fois \\ (pour notre
    version, $X=3$)
  \end{itemize}

  Quelles modifications doit-on apporter à notre stratégie gagnante pour qu'elle
  marche dans le cas général ?  }

\section*{Le coin de l'animateur}

L'objectif de cette activité est simplement d'introduire la notion d'algorithme
comme stratégie gagnante pour un problème donné.

\begin{itemize}
\item Commencez par jouer avec les participants, sans dire qu'il y a un truc. Si
  vous jouez bien, vous gagnerez à tous les coups.
\item Bien sûr, pour gagner, vous devez laisser votre adversaire commencer.
  S'il insiste pour ne pas commencer, vous pouvez toujours essayer de gagner en
  rattrapant la stratégie gagnante à la première erreur.
\item Si un participant connaît déjà la stratégie gagnante du jeu, il pourra
  vous remplacer pour jouer avec les autres participants.
\item Si vous n'êtes pas sûr d'appliquer correctement la stratégie gagnante,
  proposez un match en 3 --- ou en 5, en cas de coup dur ;) --- manches
  gagnantes.
\item Pour amener les participants à découvrir la stratégie gagnante, vous
  pouvez grouper les objets par 4, rendant ainsi l'astuce plus visible.
\end{itemize}
%%%%%%%%%%%%%%%%%%%%%%%%%%%%%%%%%%%%%%%%%%%%%%%%%%%%%%%%%%%%%%%%%%%%%%%%%
%%%%%%%%%%%%%%%%%%%%%%%%%%%%%%%%%%%%%%%%%%%%%%%%%%%%%%%%%%%%%%%%%%%%%%%%%
\chapter*{Le crêpier psycho-rigide}

À la fin de sa journée, un crêpier dispose d'une pile de crêpes désordonnée. Le
crêpier étant un peu psycho-rigide, il décide de ranger sa pile de crêpes, de la
plus grande (en bas) à la plus petite (en haut), avec le coté brûlé caché.

\begin{center}
  \input{crepier/crepier.tex}
\end{center}

Pour cette tâche, le crêpier peut faire une seule action : glisser sa spatule
entre deux crêpes et retourner le haut de la pile. Comment doit-il procéder pour
trier toute la pile ?

\begin{center}
  \input{crepier/crepier2.tex}
\end{center}

\encart{Matériel}{
  \begin{itemize}
  \item Des planchettes en bois de tailles et de couleurs différentes (faces
    reconnaissables)
  \item Une pelle à tarte pour retourner les planchettes (optionnelle)
  \end{itemize}
}

%\begin{center}
  %\includegraphics[width=0.6\linewidth]{img/crepier.pdf}
%\end{center}

\newpage

\section*{Description d'un algorithme}

L'algorithme permettant de résoudre le problème du crêpier est le suivant :

\begin{enumerate}
\item amener la plus grande crêpe en haut de la pile
\item mettre la face brûlée vers le haut
\item retourner toute la pile - la crêpe est rangée
\item recommencer en ignorant les crêpes rangées
\end{enumerate}

Cet algorithme assez simple nous apprend deux choses. Premièrement, un
algorithme n'a d'intérêt que si on peut l'expliquer - pire encore, que si on
peut l'\textit{expliquer à un ordinateur}. Il doit donc être \textbf{écrit sans
  ambiguïté}. Deuxièmement, un algorithme décompose le problème en une série de
tâches simples. On appelle ce principe \textbf{\og Diviser pour mieux
  régner\fg}.

Ce que nous venons de décrire est le c\oe{}ur de métier des informaticiens:
analyser un problème, le subdiviser en problèmes plus simples, formaliser le
tout sous la forme d'un algorithme, et traduire l'algorithme dans un langage
compréhensible par l'ordinateur.

\section*{Performance d'un algorithme}

Le premier objectif de l'écriture d'un algorithme est qu'il résolve le
problème. Le second objectif est qu'il le résolve le plus vite possible.

Évaluer la \textbf{performance d'un algorithme} nous permet de nous faire une
idée du temps nécessaire à la résolution d'un problème de grande taille. Par
exemple, combien de temps faudra-t-il à notre algorithme pour trier une pile
d'un million de crêpes ?

De plus, s'il existe plusieurs algorithmes résolvant le même problème,
l'évaluation de la performance nous donne un critère objectif pour savoir lequel
est le plus efficace.

\subsection*{Évaluer la performance d'un algorithme}

Pour évaluer la performance, on compte le nombre de \og coups \fg\ nécessaires
pour résoudre le problème dans le cas général. Pour le problème du crêpier :

\begin{itemize}
\item pour ranger une crêpe, il faut entre $0$ coup (la crêpe est déjà rangée)
  et $3$ coups (amener en haut, retourner, amener à sa place) ;
\item pour $n$ crêpes (cas général), il faut entre $0$ coup (meilleure
  situation) et $3 \times n$ coups (pire situation). La performance de
  l'algorithme dépend donc beaucoup de l'état initial, mais on s'intéresse
  surtout aux cas intermédiaires, qui sont les plus probables.
\end{itemize}

Ici, $n$ est une variable qui exprime la taille du problème. La performance d'un
algorithme est notée comme une fonction de la taille du problème nommée
$O$. Pour le crêpier, on peut donc écrire $O(3 \times n)$.

Le temps d'exécution variant selon l'état initial, la performance exprime
l'ordre de grandeur du temps d'exécution. Pour des grandes valeurs de $n$, il
n'est pas très utile de faire la distinction entre $O(n)$, $O(n+4)$ ou encore
$O(3 \times n)$ --- en particulier quand on compare avec un autre algorithme
dont la performance est $O(n^2)$.

On simplifie donc en retirant les constantes pour ne garder que les termes
importants. Pour le crêpier, on notera donc la performance de notre algorithme
$O(n)$ ; on dit alors que la performance de l'algorithme est \textit{linéaire},
car le temps de calcul croît linéairement avec $n$. Un algorithme avec une
performance $O(n^2)$ est dit \textit{quadratique}, tandis qu'un algorithme
$O(2^n)$ est dit \textit{exponentiel}.

\subsection*{À la recherche du meilleur algorithme possible}

On arrive parfois à montrer qu'on a le meilleur algorithme possible. Par
exemple, on ne peut pas trier les éléments en moins de $n$ étapes, car on doit
forcément tous les considérer.

On peut aussi prouver qu'un tri comparatif ne peut pas se faire en moins de
$n\times log(n)$ étapes, car il n'accumule pas assez d'informations pour choisir
la bonne permutation en moins d'étapes.

Mais la plupart du temps, on ne sait pas prouver que l'algorithme connu est le
meilleur possible. C'est alors le meilleur \textit{connu}, sans être forcément
le meilleur \textit{possible}.
    
\section*{Le coin de l'animateur}

L'objectif de cette activité est de trouver un algorithme, de le faire
verbaliser par les participants et d'en mesurer la performance.

\begin{itemize}
\item Expliquez les règles et demandez aux participants de tenter de résoudre le
  problème ;
\item s'ils bloquent, conseillez-les. Par exemple : \og essaye d'abord de mettre
  la grande crêpe en bas \fg, ou encore \og où doit se trouver la grande crêpe
  pour pouvoir l'amener en bas ? \fg
\item Quand les participants ont trouvé l'algorithme, demandez-leur de
  l'expliquer.
\item Demandez ensuite de calculer le nombre de coups nécessaires pour ranger la
  pile de crêpes. Le nombre de coups dépendant de l'état initial, faites-les
  généraliser en trouvant le nombre de coups maximal pour ranger une crêpe, puis
  $n$ crêpes.
\item Le discours sur le $O(n)$ est volontairement approximatif. On veut faire
  sentir les choses; faire un vrai cours prend une douzaine d'heures
  (cf. \url{http://people.irisa.fr/Martin.Quinson/Teaching/TOP/}).




  % \item Au passage, le crêpier ne ressemble pas du tout aux tours de Hanoï:
  %   l'histoire ressemble un peu, mais la résolution est très différente (il y
  %   a $2^n-1$ étapes à Hanoï et $3\times n$ au crêpier).
\end{itemize}
%%%%%%%%%%%%%%%%%%%%%%%%%%%%%%%%%%%%%%%%%%%%%%%%%%%%%%%%%%%%%%%%%%%%%%%%%
%%%%%%%%%%%%%%%%%%%%%%%%%%%%%%%%%%%%%%%%%%%%%%%%%%%%%%%%%%%%%%%%%%%%%%%%%
\chapter*{Le Base-ball multicolore}

On dispose de quatre bases de couleurs différentes, et deux joueurs associés à
chaque base. Le but du jeu est de déplacer les joueurs afin d'amener chaque
joueur sur la base correspondant à sa couleur. Il y a cependant trois
contraintes :

\begin{itemize}
\item les bases sont disposées en cercle, et un joueur ne peut se déplacer que
  vers les deux bases voisines (il ne peut pas traverser le terrain) ;
\item on ne peut déplacer qu'un joueur à la fois ;
\item chaque base a deux places, et un joueur ne peut se déplacer vers une base
  que si elle possède une place libre.
\end{itemize}

\begin{center}
  \includegraphics[width=0.3\linewidth]{baseball/baseball_coup.pdf}
  %\begin{tabular}{cc}
    %\etat{green}{blue}{red}{yellow}{white}{yellow}{green}{blue} &
    %\etat{red}{white}{green}{green}{blue}{blue}{yellow}{yellow} \\
    %État initial &
    %État final \\
  %\end{tabular}
\end{center}

\encart{Matériel}{
  \begin{itemize}
  \item Plusieurs équipes bien différenciables, chacune composée d'une base et
    de deux joueurs (des LEGO, des bouts de bois, des cailloux, du fil
    électrique de différentes couleurs, etc.)
  \item 4 équipes au minimum. On peut mettre des équipes ou des joueurs
    supplémentaires pour augmenter la difficulté.
  \end{itemize}
}

L'objectif de cette activité est d'\textbf{expliquer clairement} la méthode de
résolution du problème (algorithme), ainsi que le raisonnement qui a permis de
trouver cette méthode.

\newpage

\section*{Premier algorithme}

En suivant les règles du jeu, on observe que quelle que soit la disposition des
joueurs, 4 joueurs peuvent être déplacés vers la place vide : les deux de la
base de gauche, les deux de la base de droite. Notre algorithme sera donc une
méthode permettant de choisir à chaque étape quel coup jouer parmi ces 4
possibles.

\begin{itemize}
\item On ne s'autorise à tourner que dans un seul sens. Ainsi, le nombre de
  coups possibles descend de 4 à 2 (car 2 joueurs tourneraient à l'envers).
\item Parmi les 2 coups restants, on déplace le joueur qui a la plus grande
  distance à parcourir avant d'arriver à sa base (Si la distance est la même,
  c'est que les deux joueurs ont la même couleur - les deux coups sont alors
  équivalents).
\item Tant que tous les joueurs ne sont pas rentrés à leur base, on continue les
  déplacements.
\end{itemize}

% FIXME tikz ?
\begin{center}
  \includegraphics[width=\linewidth]{baseball/baseball_ex1.pdf}
\end{center}

Ici, nous n'avons représenté que les 4 premières étapes, mais l'algorithme
arrive à la solution en 15 étapes.

\subsection*{Regardons plus en détail{\ldots}}

À première vue, cet algorithme est attirant : il est assez simple et semble
relativement rapide --- 15 coups pour 4 bases et 7 joueurs --- ce n'est pas si
mal. Pourtant, il y a un problème : \textit{cet algorithme est faux}.

Pour s'en convaincre, il suffit de prendre le jeu dans son état résolu, et
d'intervertir deux joueurs. On observe alors que notre algorithme ramène le jeu
à son état initial sans atteindre la solution - notre algorithme boucle donc à
l'infini.

% FIXME tikz ?
\begin{center}
  \includegraphics[width=\linewidth]{baseball/baseball_ex2.pdf}
\end{center}

\newpage

\section*{Deuxième algorithme}

Notre premier algorithme étant faux, réfléchissons à un autre algorithme. 

Commençons par \textit{apprendre de nos échecs}. Notre premier algorithme boucle
parfois à l'infini. Pour réparer cela, on pourrait s'interdire de faire un tour
complet en coupant le cercle. Pour ne pas se tromper, cela revient à disposer
les bases sur une ligne.

% FIXME tikz ?
\begin{center}
  \includegraphics[width=\linewidth]{baseball/baseball_ligne.pdf}
\end{center}

Puis, \textit{apprenons de nos réussites}. Dans le problème du crêpier, on a
résolu le problème en réduisant progressivement sa taille : on place la première
crêpe, puis la deuxième etc. On s'est fixé des objectifs intermédiaires qui
décomposent le problème en étapes plus simples.

À partir de ces enseignements, on peut construire l'algorithme suivant :

% FIXME formulation complêtement merdique ! un ordi ne pourra pas comprendre ça ...
\begin{itemize}
\item On coupe le cercle de sorte que la base avec la place vide soit à
  l'extrémité droite.
\item On s'occupe des bases les unes après les autres, de gauche à droite.
\item Pour rapprocher un joueur de sa base, on déplace les joueurs des autres
  couleurs pour amener le trou à gauche du joueur à déplacer.
\item On répète l'opération jusqu'à ce que les deux joueurs soient revenus à
  leur base, et on n'y touche plus.
\end{itemize}

% FIXME tikz ? l'illustration est fausse, la base rouge devrait être à droite.
\begin{center}
  \includegraphics[width=\linewidth]{baseball/baseball_ex3.pdf}
\end{center}

On peut maintenant ignorer la première base, qui est déjà rentrée.

\section*{Correction d'algorithmes}

Notre deuxième algorithme est un peu plus complexe que le précédent, mais il est
correct --- une propriété très importante quand on écrit un algorithme. Mais
comment être sûr de la \textbf{correction} d'un algorithme ? Il existe plusieurs
approches :

\begin{description}
\item[Tester tous les cas.] On vérifie que l'algorithme trouve la solution pour
  tous les cas possibles. Mais ça ne marche que pour un problème de taille
  finie, et montrer de cette manière que notre algorithme est correct pour 4
  bases n'implique pas sa correction pour 5 bases.
\item[Écrire une preuve mathématique.] On peut prouver mathématiquement que cet
  algorithme fonctionne pour le cas général $m$ bases. Ce n'est pas trivial,
  mais les chercheurs en informatique en ont écrit des plus difficiles.
\item[Ça ressemble à un algorithme classique.] On peut observer une certaine
  ressemblance avec un algorithme déjà connu. Mais ça ne prouve rien, au
  fond{\ldots}
\end{description}

\subsection*{Les algorithmes classiques}

Les informaticiens apprennent par c\oe{}ur des algorithmes (abstraits) à
l'école. Face à un problème nouveau, un informaticien cherche généralement à le
ramener à un problème connu. Ce rapprochement se fait en trouvant des analogies
ou en décomposant le problème en plusieurs sous-problèmes connus.

Par exemple, quand des collègues informaticiens jouent au crêpier, ils demandent
avant tout si c'est \og une tour de Hanoï \fg, jeu bien connu des
informaticiens. Pour le base-ball multicolore, notre algorithme ressemble à un
\og tri à bulle \fg, autre algorithme bien connu. Mais cette ressemblance ne
suffit pas à prouver la correction de notre algorithme. Pour la prouver, il
faudrait démontrer que notre algorithme est un cas particulier du tri à bulle.
 
\subsection*{Les algorithmes de tri}

Les algorithmes de tri sont très classiques en informatique. D'une part, ils
permettent d'expliquer les grands principes aux élèves (\og diviser pour régner
\fg, récursivité, algorithmes gloutons {\ldots}). D'autre part, les ordinateurs
trient très souvent des données, car beaucoup de problèmes sont plus simples
après (trouver un livre est plus simple dans une bibliothèque rangée, par
exemple). \textit{Les musiciens font leurs gammes, les informaticiens débutants
  apprennent leurs algorithmes}.

%\subsection*{Le travail des chercheurs}

%Une partie du travail des chercheurs en informatique est d'améliorer les
% algorithmes connus, d'en inventer de nouveaux et de démontrer leur correction,
% d'en comparer les performances ...

\section*{Le coin de l'animateur}

L'objectif de cette activité est d'introduire la notion de correction
d'algorithme.

\begin{itemize}
\item Laissez les participants chercher un peu en les faisant verbaliser.
\item S'ils sont sur le point de trouver l'algo juste, introduisez très vite
  l'algo faux pour préserver un enchaînement logique~: \og oui, ok, mais je vais
  vous montrer une façon de faire rigolote. \fg
\item Quand l'algo juste est établi, et avant de parler de performance, on peut
  partir sur une variante :
  \begin{itemize}
  \item Chaque participant prend une couleur (une base placée au sol entre ses
    pieds)
  \item Chaque participant (sauf 1) prend un bonhomme dans chaque main
  \item À chaque étape, celui qui a une main libre prend un bonhomme dans la
    main d'un voisin
  \item (Attention, c'est fastidieux à 8 ou 9 couleurs, il vaut mieux faire deux
    rondes car l'algorithme semble $O(n^2)$)
  \end{itemize}
\item Expérimentalement, l'algorithme qui tourne converge très souvent vers la
  solution à 5 bases, mais converge souvent vers la boucle infinie quand il y a
  plus de couleurs. Ne tentez pas le diable ;)
\item Dans la disposition linéaire, il est plus simple de mettre la couleur avec
  un seul bonhomme à une extrémité, et commencer par remplir la maison de
  l'autre extrémité. Sinon, on se retrouve avec une maison remplie de un seul au
  milieu, et il faut comprendre que la solution passe par le stockage temporaire
  d'un pion de la maison d'à coté sur le trou.
\item Il serait intéressant de prouver effectivement la correction de
  l'algorithme linéaire, ainsi que de quantifier la probabilité de
  fonctionnement de l'algo qui tourne en fonction du nombre de maisons.
\end{itemize}
\end{document}
